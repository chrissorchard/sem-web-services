\section{Introduction}

Web services are software entities that allow programmatic interaction between
hosts on the web, using technologies such as HTTP and XML and known design patterns and
frameworks to create systems that are interoperable and reusable. There
exists
two approaches to web services that are currently in widespread use: those based
on a pair W3C standards called Web Service Description Language (WSDL) and the
Simple Object Access Protocol (SOAP), and those based on the 
Representational State Transfer (REST) architecture developed by the W3C and Roy
Fielding alongside the HTTP standard. Web services that use the REST design
principles are said to be RESTful.

Semantic web services are an extension of the concept of web services that
attempt to overcome one of the major shortcomings of regular web services;
whilst in WSDL based web services the syntactics of the web service are well
defined, the operations and their arguments and message formats, there is
little to no semantic data meaning that the integration and
composition of different web services to create a more complex service must be
carried out by hand. A similar problem exists with RESTful web services but is
compounded by a general lack of syntactic specification. Semantic web services
aim to describe the semantics of the operation of the underlying or contained web
service in a manner in which can be reasoned on, and as such automate the
composition and execution of complex web services\cite{mcilraith_semantic_2001}.

Within semantic web services, two different approaches have emerged for
semantically describing web services. The first involves describing the entire
web service and modelling its functionality in using a rules language, allowing
for completely automated composition and execution of web services. The second
is a more lightweight approach that developed after it became apparent that the
more involved approach was not being widely adopted\cite{battle_report_????},
and is based on annotating web service descriptions with semantic information,
whilst still allowing for functional description of the web
service\cite{kopecky_sawsdl:_2007}.

\section{A Traditional Approach}

One of the first efforts to try and apply principles of the semantic web to web
services was DAML-S, first released in 2001. DAML-S was developed with the
intention of creating a markup language capable of describing all of the facets
of a web service enabling automated discovery, composition and execution of a
web service. McIlraith et al. illustrate this principle by building an agent in
a variant of prolog that can perform automated service composition to build a
travel booking agent\cite{mcilraith_semantic_2001}. DAML-S was developed into
OWL-S on the release of OWL, extending OWL with a rules language to allow
the description of the functionality of web services and allowing the semantics
of the web service to be expressed in RDF/XML. OWL-S was submitted to the W3C in
2004, but has not been made a recommendation.

An alternative semantic web services framework is the Web Service Modelling
Ontology (WSMO) submitted to the W3C in 2005, that takes roughly the same
approach as OWL-S, but focuses on the mediation of data and concepts between
web services, translating between the ontologies, web service descriptions, and
user constraints that make up WSMO. WSMO can be expressed using the Web Service
Modelling Language (WSML), developed specifically to express the concepts in WSMO and
provide a framework for the formal logical description of a service. An
execution environment (WSMX) also exists as a reference implementation of WSML,
capable of automatic composition and execution.

When OWL-S and WSMO were created, WSDLs were the principal way to implement web
services, and both frameworks were designed to allow integration with WSDL based
web services\cite{martin_bringing_2007-1}\cite{roman_web_2005}. 
The process of translating between the semantics defined
in OWL-S or WSMO and the syntactics of WSDLs is known as grounding. In
OWL-S and WSMO, WSDLs are linked into the semantic descriptions as extra
properties on the semantic web service. There are two types of grounding
required, one for converting between the semantically described data being
exchanged, and one to convert the description of the behaviour of the service to
its interfaces. Kopecky et al. explore the grounding mechanisms of both OWL-S
and WSMO and find that whilst data grounding is fairly straightforward, the
behavioural grounding is very complex in the case of WSMO, and whilst simpler in
OWL-S, still requires the use of AI techniques\cite{kopecky_semantic_2006}.

It is useful to re-state the original goals of semantic web services, that is to add
semantic descriptions to web services to allow automatic discovery, execution
and composition of services. Whilst the above technologies are capable of
achieving this, they do so by describing the operation of the web service in
full using a rule language. The formal modelling of programs was identified as an area of
ongoing research in a 2005 W3C workshop on semantic web services, and the
workshop concludes that this has contributed to the poor adoption of these 
technologies\cite{battle_report_????}. To define a more achievable goal 
the workshop recommends a compromise to
allow work on using semantic web services for discovery and composition of web
services to be standardised whilst allowing for continued research into using
rule languages for automated composition and execution.

\section{Annotation: A Lightweight Approach}

Having found some limitations with the usability of the fully featured semantic
web services, the W3C
started work on a more lightweight approach to semantic web services. Based on
the WSDL-S submission that proposed creating semantic web services by annotating
WSDL files, Semantic Annotations for WSDL and XML Schema (SAWSDL) became a W3C
recommendation in 2007. SAWSDL does not specify an ontology for describing the
web services it annotates, but provides the \verb=modelReference= annotation that allows the
implementer to pick an ontology such as those used as a part of  WSMO or OWL-S.
Data transformation is
provided by the \verb=liftingSchemaMapping= and \verb=loweringSchemaMapping=
annotations that specify how to transform the XML used in the web service to
RDF and back. Martin et al. write that from an OWL-S viewpoint SAWSDL is not an
ideal solution to semantic web services and that it is rather poorly defined
in terms of specifying descriptions of web services, however they also find that
SAWSDL can be integrated with OWL-S descriptions\cite{martin_bringing_2007}.

Recently, the RESTful approach to web services has become more popular and
developments in semantic web services have reflected this. The annotation
approach is much more suited to RESTful web services, as they do not typically
have a contracted specification like a WSDL, so any description or documentation
of their functionality is typically in HTML\cite{kopecky_hrests:_2008}. To
annotate the HTML documentation with the semantic description of the web
service Kopeky et al. propose hRESTS, a
microformat\footnote[1]{microformats.org} for describing RESTful web service\cite{kopecky_hrests:_2008}.
hRESTS describes the web service in a manner similar to a WSDL, with
classifications for service, operations, and messages. 
SA-REST, submitted to the W3C in 2010, builds upon this by providing semantic
annotations in the style of SAWSDL to allow lifting and lowering, and to allow
the addition of OWL-S or WSMO functional descriptions\cite{lathem_sa-rest_2007}.
SA-REST also supports the use of RDFa and GRDDL as annotation
formats\cite{sheth_semantics_2008}.

Whilst some in the research community view the loosely defined annotation based approach
to semantic web services typified by SAWSDL and its RESTful derivatives as
much weaker than is necessary to achieve the goals of semantic web services
\cite{martin_bringing_2007}, others are moving in different direction, towards a
level of semantic description that is intended to allow web developers to easily discover
and combine web services to produce new services, without requiring any in-depth
knowledge of topics such as rule languages \cite{kopecky_hrests:_2008}\cite{lathem_sa-rest_2007}
\cite{sheth_semantics_2008}. 

%Both SAWSDL and SA-REST reduce the barrier to entry
%of creating semantic web services from the level of knowledge required to model
%services in WSMO or OWL-S\cite{}.

\section{Conclusions and Future Developments}

%re-write to reflect change in needs of SWS
Semantic web services have developed significantly since their inception, but
are not yet widely deployed to the point where it is possible
to use a semantic client to automatically discover, compose, and execute web
services using semantic descriptions and a semantically enabled client. Whilst the
release of SAWSDL as a W3C standard marks progress in the field of semantic web
services, it marks a step back from the set of features initially envisioned by
the creator of the DAML-S and WSMO technologies. However, the technologies
employed by web services have also changed since DAML-S and WSMO were developed, 
with a major shift towards RESTful web services. New models such as SA-REST 
have emerged to cater for this trend and provide a lighter approach
to semantic web services. It is hoped that this leads to a greater adoption of
semantic web services, in a way that perhaps more akin to how open data is being
adopted\cite{bizer_linked_2009}.


