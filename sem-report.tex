\documentclass{acm_proc_article-sp}

%some packages, mostly from malucrawl report
%\usepackage[pdftex]{graphicx}
\usepackage[utf8x]{inputenc}
\usepackage{cite}
%\usepackage{hyperref}
%\usepackage{listings}
%\usepackage{nomencl}

\begin{document}

\title{Semantic Web Services: A Practical View}

\numberofauthors{1}
\author{
\alignauthor
%Chris Orchard\titlenote{An Undergratuate Masters student in Computer Science}\\
Chris Orchard\\
    \affaddr{University of Southampton}\\
    \affaddr{University Road}\\
    \affaddr{Southampton SO17 1BJ. UK}\\
    \email{cso1g09@ecs.soton.ac.uk}
}

\maketitle
\begin{abstract}

Semantic web services are an extension on the existing concept of web services,
providing a mechanism to semantically descripe the operation and characteristics
of a web service, with the aim of allowing automated discovery, composition and
execution of web services. This report explores the progress made towards this
aim, and finds that the semantic web services community have moved away from
this goal of automation towards a more lightweight approach based on the
annotation of services. The motivations behind this move are investigated and
recent developments with RESTful semantic web services are examined, although it
is not clear to what extent the more recent implementations have been deployed.

\end{abstract}

%TODO: need classification here?

\keywords{Semantic web, web services, RESTful, WSDL}

\section{Introduction}

Web Services are a class of software for interaction between machines over the
web, using web technologies such as HTTP and XML and known design patterns and
frameworks to create systems that allow interoperability and reuse. There exist
two approaches to web services that are currently in widespread use: those based
on a pair W3C standards called Web Service Description Language (WSDL) and the
Simple Object Access Protocol (SOAP), and those based on the 
Representational State Transfer (REST) architecture developed by the W3C and Roy
Fielding alongside the HTTP standard. Web services that use the REST design
principles are said to be RESTful.

Semantic web services are an extension of the concept of web services that
attempt to overcome one of the major shortcomings of regular web services, that
is, whilst in WSDL based web services the syntactics of the web service are well
defined, the operations and their arguments and message formats, but there is
little to no semantic data, something that means that the integration and
composition of different web services to create a more complex service must be
carried out by hand. A similar problem exists with RESTful web services, but is
compounded by a general lack of syntactic specification. Semantic web services
aim to describe the semantics of the operation of the underlying/contained  web
service in a manner in which can be reasoned on, and as such automate the
composition and execution of complex web services(cite).

Semantic web services evolved from the efforts into developing WSDL and SOAP
based web services by the W3C, and only later began to cater for RESTful web
services as the REST paradigm became a popular option for implementing web APIs.
This report attempts to collect the latest developments in semantic web services
of both styles, and then compare these approaches to explore where semantic web
services might fit into the future of web services.

\section{A Traditional Approach}

One of the first efforts to try and apply principles of the semantic web to web
services was DAML-S, first released in 2001. DAML-S was developed with the
intention of creating a markup language capable of describing all of the facets
of a web service, enabling automated discovery, composition and execution of a
web service. McIlraith et al. illustrate this principle by building an agent in
a variant of prolog that can perform automated service composition to build a
travel booking agent\cite{mcilraith_semantic_2001}. DAML-S was developed into
OWL-S on the release of OWL, extending OWL with a rules language to allow
the description of the functionality of web services and allowing the semantics
of the web service to be expressed in RDF/XML. OWL-S was submitted to the W3C in
2004, but has not been made a recommendation.

An alternative semantic web services framework is the Web Service Modelling
Ontology (WSMO) submitted to the W3C in 2005, that takes roughly the same
approach as OWL-S, but focuses on the mediation of data and concepts between
web services, translating between the ontologies, web service descriptions, and
user constraints that make up WSMO. WSMO can be expresses using the Web Service
Modelling Language, developed specifically to express the concepts in WSMO and
provide a framework for the formal logical description of a service. An
execution environment (WSMX) also exists as a reference implementation of WSML,
capable of automatic composition and execution.

When OWL-S and WSMO were created, WSDLs were the principal way to implement web
services, and both frameworks were designed to allow integration with WSDL based
web services(cite?). The process of translating between the semantics defined
in OWL-S or WSMO and the syntactics of WSDLs is known as grounding(cite). In
OWL-S and WSMO, WSDLs are linked into the semantic descriptions as extra
properties on the semantic web service. There are two types of grounding
required, one for converting between the semantically described data being
exchanged, and one to convert the description of the behaviour of the service to
its interfaces. Kopecky et al. explore the grounding mechanisms of both OWL-S
and WSMO and find that whilst data grounding is fairly straightforward, the
behavioural grounding is very complex in the case of WSMO, and whilst simpler in
OWL-S, still requires the use of AI techniques(cite).

%TODO: this paragraph is wrong and needs fixing
It is useful to re-state the goals of semantic web services, that is to add
semantic descriptions to web services to allow automatic discovery, execution
and composition of services. Whilst the above technologies achieve this, they do
so by creating all-encompassing frameworks, and whilst it is supposedly
possible to build semantic web services in this fashion around existing web
services(cite), building ontologies and logic that accurately models the entire
service is an activity which may involve significant effort. This also runs
against the "contract first" design principles of WSDL web services, intended to
improve the interoperability of web services by first specifying the interface
of the service(cite). A 2005 W3C workshop on semantic web services concluded that the
"ambitious, comprehensive vision pursued by many research efforts" were a
possible cause of the lack of adoption (and lack of progress towards
standardisation) of these technologies(cite). 


\section{Annotation: A Lightweight Approach}

Given the issues with the heavyweight semantic web service frameworks, the W3C
started work on a more lightweight approach to semantic web services. Based on
the WSDL-S submission that proposes creating semantic web services by annotating
WSDL files, Semantic Annotations for WSDL and XML Schema (SAWSDL) became a W3C
recommendation in 2007.  

\section{Conclusions and Future Developments}



\bibliographystyle{abbrv}
\bibliography{sem-report}

\end{document}
