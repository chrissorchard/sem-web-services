\section{Introduction}

Web Services are a class of software for interaction between machines over the
web, using web technologies such as HTTP and XML and known design patterns and
frameworks to create systems that allow interoperability and reuse. There exist
two approaches to web services that are currently in widespread use: those based
on a pair W3C standards called Web Service Description Language (WSDL) and the
Simple Object Access Protocol (SOAP), and those based on the 
Representational State Transfer (REST) architecture developed by the W3C and Roy
Fielding alongside the HTTP standard. Web services that use the REST design
principles are said to be RESTful.

Semantic web services are an extension of the concept of web services that
attempt to overcome one of the major shortcomings of regular web services, that
is, whilst in WSDL based web services the syntactics of the web service are well
defined, the operations and their arguments and message formats, but there is
little to no semantic data, something that means that the integration and
composition of different web services to create a more complex service must be
carried out by hand. A similar problem exists with RESTful web services, but is
compounded by a general lack of syntactic specification. Semantic web services
aim to describe the semantics of the operation of the underlying/contained  web
service in a manner in which can be reasoned on, and as such automate the
composition and execution of complex web services(cite).

Semantic web services evolved from the efforts into developing WSDL and SOAP
based web services by the W3C, and only later began to cater for RESTful web
services as the REST paradigm became a popular option for implementing web APIs.
This report attempts to collect the latest developments in semantic web services
of both styles, and then compare these approaches to explore where semantic web
services might fit into the future of web services.

\section{WSDL-Based Approaches}



\section{RESTful Approaches}

\section{Comparing Approaches}

\section{Conclusions and Future Developments}

Some awesome report\cite{durand_deploying_2001}.
